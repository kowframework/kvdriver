this is the bas tex
% APQ-2.2 Manual

\documentclass[portuguese,letterpaper]{book}
\usepackage{times}
\usepackage[T1]{fontenc}
\usepackage[utf8]{inputenc}
\usepackage{ucs}
\usepackage{longtable}
\usepackage{amsmath}
\usepackage{amssymb}
\usepackage{fancyhdr}

\pagestyle{fancy}

%\lhead{}
%\chead{}
\rhead{}

\lfoot{}
\cfoot{\thepage}
\rfoot{Metarrevisor KOW}


\newcommand\Ref[1]{\textsection\ref{#1} (page~\pageref{#1})}

\usepackage{fancyvrb}
\usepackage{listings}

\usepackage{makeidx}
\makeindex

\IfFileExists{url.sty}{\usepackage{url}}
                      {\newcommand{\url}{\texttt}}

\makeatletter

\usepackage{babel}
\makeatother



%==========%
% HYPERREF %
%==========%
\usepackage[dvipdfm, bookmarks, colorlinks, breaklinks, pdftitle={Relatório Gerencial KOW},
    pdfauthor={Metarrevisor KOW}]{hyperref}
\hypersetup{
	linkcolor=DarkSkyBlue,
	citecolor= DarkSkyBlue,
	filecolor= DarkSkyBlue,
	urlcolor= DarkSkyBlue
}



%========================%
% Listings Package Setup %
%========================%
\usepackage{xcolor}
\usepackage{listings}


% COLORS (Tango)
\definecolor{LightButter}{rgb}{0.98,0.91,0.31}
\definecolor{LightOrange}{rgb}{0.98,0.68,0.24}
\definecolor{LightChocolate}{rgb}{0.91,0.72,0.43}
\definecolor{LightChameleon}{rgb}{0.54,0.88,0.20}
\definecolor{LightSkyBlue}{rgb}{0.45,0.62,0.81}
\definecolor{LightPlum}{rgb}{0.68,0.50,0.66}
\definecolor{LightScarletRed}{rgb}{0.93,0.16,0.16}
\definecolor{Butter}{rgb}{0.93,0.86,0.25}
\definecolor{Orange}{rgb}{0.96,0.47,0.00}
\definecolor{Chocolate}{rgb}{0.75,0.49,0.07}
\definecolor{Chameleon}{rgb}{0.45,0.82,0.09}
\definecolor{SkyBlue}{rgb}{0.20,0.39,0.64}
\definecolor{Plum}{rgb}{0.46,0.31,0.48}
\definecolor{ScarletRed}{rgb}{0.80,0.00,0.00}
\definecolor{DarkButter}{rgb}{0.77,0.62,0.00}
\definecolor{DarkOrange}{rgb}{0.80,0.36,0.00}
\definecolor{DarkChocolate}{rgb}{0.56,0.35,0.01}
\definecolor{DarkChameleon}{rgb}{0.30,0.60,0.02}
\definecolor{DarkSkyBlue}{rgb}{0.12,0.29,0.53}
\definecolor{DarkPlum}{rgb}{0.36,0.21,0.40}
\definecolor{DarkScarletRed}{rgb}{0.64,0.00,0.00}
\definecolor{Aluminium1}{rgb}{0.93,0.93,0.92}
\definecolor{Aluminium2}{rgb}{0.82,0.84,0.81}
\definecolor{Aluminium3}{rgb}{0.73,0.74,0.71}
\definecolor{Aluminium4}{rgb}{0.53,0.54,0.52}
\definecolor{Aluminium5}{rgb}{0.33,0.34,0.32}
\definecolor{Aluminium6}{rgb}{0.18,0.20,0.21}




%textsection\ref{#1} (page~\pageref{#1})}

\newcommand\money[1]{R\$ #1}



%==============%
% The document %
%==============%


\begin{document}

\title{Relatório Metarrevisor KOW}
\author{%
Gerado automaticamente pelo Metarrevisor KOW
}
\date{\today}
\maketitle

\tableofcontents{}

\chapter{Introduction}
Esse relatório foi criado automaticamente e contém todas as modificações operadas 
pelo Metarrevisor KOW em projetos criados pelo seu usuário.


\section{Resumo}
\begin{tabular}{|l@{ }c|}
	\hline
	Páginas processadas	& @_total_pages_@ \\
	\hline
	Custo total		& \money{@_total_price_@} \\
	\hline
\end{tabular}


\section{Tipos de Operação}

\subsection{TROCA}
	Essa operação significa que há a recomendação de trocar as palavras para
	adequação ao novo Acordo Ortográfico.\\

	O seu processamento é por regras, que podem ser definidas em uma base de
	dados. Novas regras podem ser encomendadas se for preciso.

\subsection{ERRO}
	Significa que aconteceu algum problema no lado do servidor. Esses problemas
	devem ser reportados à equipe técnica da KOW assim que encontrados.\\

	Essa variante de operação fornece ao sistema maior robustez e estabilidade,
	pois permite que o usuário processe ao menos parte de seus arquivos e
	defeitos na implementação não atrapalham no processamento de arquivos
	novos.\\

	Ao reportar erro, favor encaminhar junto o arquivo original que apresentou
	problema e a mensagem que aparece nesse relatório.



\chapter{Log}
@@TABLE@@

\section{@_project_name_@}
	@@TABLE@@
	@@IF@@ @_log_date_@ /= ""
	\subsection{@_log_date_@}
		Arquivo processado pelo usuário \emph{@_log_user_identity_@}.\\

		\begin{tabular}{|l@{ }c|}
			\hline
			Catálogo usado		& @_catalog_name_@ \\
			Custo por página	& \money{@_catalog_pp_price_@} \\
			\hline
			Páginas processadas	& @_log_pages_@ \\
			Custo total		& \money{@_log_price_@} \\
			\hline
		\end{tabular}

		Informações sobre os programas:

		\begin{tabular}{|l@{ }l|}
			\hline
			Extrator	&  @_contents_generator_@ \\
			Processador	& @_commands_generator_@ \\
			\hline
		\end{tabular}

	
		\subsubsection{Trocas}
			\begin{tabular}{|c c l l|}
				\hline
				Sessão		& Subsessão	& Padrão Encontrado	& Substituição \\
				\hline
				@@TABLE@@
					@_command_troca_section_@	& 
					@_command_troca_subsection_@	& 
					@_command_troca_param1_@	& 
					@_command_troca_param2_@ \\
				@@END_TABLE@@
				\hline
			\end{tabular}

		\subsubsection{Erros}
			\begin{tabular}{|c c l l|}
				\hline
				Sessão		& Subsessão	& Tipo do erro	& Detalhes  \\
				\hline
				@@TABLE@@
					@_command_erro_section_@	& 
					@_command_erro_subsection_@	& 
					@_command_erro_param1_@	& 
					@_command_erro_param2_@ \\
				@@END_TABLE@@
				\hline
			\end{tabular}
	

	@@END_IF@@
	@@END_TABLE@@
@@END_TABLE@@


\end{document}
