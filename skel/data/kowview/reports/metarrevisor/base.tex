
\documentclass{article}
\usepackage{kow} % Define tudo que vc precisa para sempre!


% Comandos:								 
   \newcommand\money[1]{R\$ #1}		% Dinheiro
   \defineshorthand{"/}{\discretionary{}{/}{/}} 


\title{\Huge \textls{\textsc{metarrevisor}}\footnote{Desenvolvido pela Kow Software Livre e Consultoria em parceria com as editoras Esfera e Hedra}}

\begin{document}

\maketitle


\begin{abstract}
Esse relatório foi criado automaticamente e contém todas as modificações 
que o \textsc{Metarrevisor} executou em seus projetos. 
As duas operações básicas indicadas são as \textit{trocas}: substituições
de palavras  conforme o Novo Vocabuário Ortográfica, e os \textit{erros}:
mensagens sobre problemas no servidor, e que você deve nos relatar
pelo e-mail metarrevisor@kow.com.br. Ao reportar um erro,  encaminhe por favor o 
arquivo original que apresentou o problema e a mensagem que aparece nesse relatório.
\end{abstract}


\setcounter{tocdepth}{2}            % amplitude da presença das partes no índice
\setcounter{secnumdepth}{1}   % amplitude da numeração das partes  
\tableofcontents

\section{Instruções}

\begin{center}
\begin{xtabular}{lr}
	\hline
	Páginas processadas	&  424 \\
	\hline
	Custo total		& \money{ 636.00} \\
	\hline
\end{xtabular}
\end{center}

\section{Arquivos processados}



% GAMBI FIX ::
% O templates parser do AWS faz bagunça ao acessar variáveis de vetores dentro de um looping para matrizes
% isso resolve o problema
\newcommand\projectname[0]{nada}


@@TABLE@@
	\renewcommand\projectname[0]{@_project_name_@}

	@@TABLE@@

	@@IF@@ @_log_date_@ /= ""
	\subsection{@_resource_name_@ (@_log_date_@)}

			\tablehead{
				\multicolumn{1}{l}{}	&
				\multicolumn{1}{l}{}	\\ 
				}

		\begin{center}
			\begin{xtabular}{rp{.6\textwidth}}
				usuário			& @_log_user_identity_@ \\
				projeto			& \projectname \\
				catálogo		& @_catalog_name_@ \\
				custo página		& \money{@_catalog_pp_price_@} \\
				páginas processadas	& @_log_pages_@ \\
				custo total		& \money{@_log_price_@} \\
				cliente			& @_contents_generator_@ \\
				servidor		& @_commands_generator_@ 
			\end{xtabular}
		\end{center}


		% TABELA %%%%%%%%%%%%%%%%%%%%%%%%%%%%%%%%
		\subsubsection{\textsc{Trocas}}
		
			\tablehead{
				\multicolumn{1}{l}{Sessão}	&
				\multicolumn{1}{l}{Subsessão}		& 
				\multicolumn{1}{l}{Padrão Encontrado}		& 
				\multicolumn{1}{l}{Substituição}	\\ \hline\hline	
				}
		
			\begin{center}
				\begin{xtabular}{l | l | p{.3\textwidth} | p{.3\textwidth}}
					@@TABLE@@
						@@IF@@ @_command_troca_section_@ /= ""
							@_command_troca_section_@	& 
							@_command_troca_subsection_@	& 
							@_command_troca_param1_@	& 
							@_command_troca_param2_@ \\
						@@END_IF@@
					@@END_TABLE@@
				\end{xtabular}
			\end{center}

	
		\subsubsection{\textsc{Erros}}
		
			\tablehead{
				\multicolumn{1}{l}{Sessão}		&
				\multicolumn{1}{l}{Subsessão}		& 
				\multicolumn{1}{l}{Erro}		& 
				\multicolumn{1}{l}{Detalhes}	\\ \hline\hline	
				}
		
			\begin{center}
				\begin{xtabular}{l | l | p{.3\textwidth} | p{.3\textwidth}}
					@@TABLE@@
						@@IF@@ @_command_erro_section_@ /= ""
							@_command_erro_section_@	& 
							@_command_erro_subsection_@	& 
							@_command_erro_param1_@	& 
							@_command_erro_param2_@ \\
						@@END_IF@@
					@@END_TABLE@@
				\end{xtabular}
			\end{center}
	

	@@END_IF@@
	@@END_TABLE@@



@@END_TABLE@@




\noindent\dotfill

\end{document}
